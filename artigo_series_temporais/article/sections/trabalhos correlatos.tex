\section{Trabalhos Correlatos}

Historicamente, modelos lineares têm dominado previsões de séries temporais 
pela sua simplicidade de implementação, baixo custo computacional e facilidade 
para entender seu funcionamento, o que faz com que eles sejam bem conhecidos 
e efetivos na resolução de uma ampla gama de problemas, como pode ser visto 
em diferentes estudos que envolvem crescimento populacional \cite{Alan:5}. Apesar disso, 
contextos específicos trazem uma demanda por ferramentas mais robustas, baseadas 
em diferentes algoritmos. Dentro das técnicas mais modernas com \emph{Deep Learning}, 
são notáveis o uso de \emph{Convolutional Neural Networks} \cite{Ban:3} e \emph{Long Short-Term Memory 
Networks} \cite{Benjamin:4} aplicadas a séries temporais. Esses modelos são originalmente voltados 
a análises multivariadas com bases de dados massivas, o que faz com que as suas 
características mais notáveis sejam a alta resistência a ruídos na base de dados 
e a habilidade de aprender e extrair automaticamente as características principais
de uma série. Apesar de apresentarem grandes vantagens, para funcionar corretamente 
e com alto grau de confiabilidade, também é demandada uma base de dados extensiva, 
além do custo computacional ser muito maior. Essa característica é levada em 
consideração em diferentes contextos, inclusive em problemas que não possuem 
uma base de dados extensa o bastante. Este ponto faz com que o presente trabalho 
busque evitar tal barreira, visto que nem sempre cidades e entidades governamentais 
possuem tal base de dados capaz de suportar um modelo de \emph{Deep Learning}. Com isso 
em mente, o presente trabalho apresenta uma busca por um modelo de \emph{Machine Learning} 
capaz de realizar predições de medidores populacionais apenas com dados passados, 
em forma de séries temporais, já que é entendido que para previsões de médio prazo, 
que são previsões de 10 anos no futuro, não são necessárias tais detalhamentos.

