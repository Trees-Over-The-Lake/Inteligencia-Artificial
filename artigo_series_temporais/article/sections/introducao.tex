\section{Introdução}

O acelerado crescimento populacional e econômico experienciado por inúmeros 
países no século XX demandou uma expansão da capacidade de processamento e 
interpretação de informações em diferentes setores da sociedade, com uma das 
técnicas mais relevantes sendo a análise de séries temporais devido à sua ampla 
aplicação em diferentes setores da sociedade. Séries temporais são definidas 
como uma coleção de observações expandidas ao longo de um determinado período 
de tempo, e sua representação consiste em uma série de pontos indexados e 
ordenados ao longo de um intervalo constante de tempo \cite{BOX:1}. Tendo isso em vista, 
diferentes conjuntos de dados podem ser entendidos e representados como séries 
temporais: a quantidade de objetos produzidos por uma fábrica, uma representação 
numérica do número semanal de acidentes em uma estrada, precipitação ao longo 
de um ano, observações de hora em hora acerca de um processo químico, e 
diversos outros. Exemplos de usos de séries temporais também podem ser encontrados 
em setores como finanças, geografia, engenharia, ciências naturais e ciências sociais \cite{WEIGEND:2}.

Tendo isto em vista, é natural que séries temporais se tornem cada vez mais presentes
no contexto de análise de dados a fim de diversificar o número de técnicas utilizadas 
e prover alternativas mais adequadas a cada cenário. O presente trabalho visa explorar 
esse processo ao expor cenários onde séries temporais podem representar fenômenos e 
formas de interpretação de dados.

A análise de séries temporais é uma categoria de técnicas capazes de produzir um 
modelo que leva em consideração dados passados da série para fazer uma predição do 
valor de um dado desconhecido. Estas técnicas são úteis à medida que estimar 
valores relevantes em determinados contextos pode se provar imperativo para 
tomadas de decisões bem-sucedidas, com essas previsões podendo ser aplicadas 
nos mais diferentes contextos e cenários, como explicitado anteriormente. 
Adicionalmente, como os modelos prevêem novos valores utilizando como base 
valores passados, é importante que haja coerência na ordem de observação a fim de 
possibilitar que o modelo note a presença de características e padrões na série. 
Esta característica pode ser aproveitada para mapear e mensurar diferentes 
métricas que estão ordenadas ao longo de um período de tempo. Com isto em mente, 
o presente artigo tem como objetivo utilizar a análise de séries temporais para 
avaliar a evolução de diferentes medidores populacionais brasileiros de 1960 a 2020.

