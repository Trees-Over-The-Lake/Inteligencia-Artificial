\section{Conclusão}

Ao final do projeto, foram obtidas previsões para três diferentes indicadores sociais 
brasileiros nos anos de 2011 a 2020 com pouco nível de desvio, o que implica na possibilidade 
do uso do modelo desenvolvido para prever os valores populacionais futuros no Brasil e 
auxiliar a tomada de decisões governamentais em nível federal, estadual e municipal. É 
relevante notar como as três métricas apresentaram excelentes resultados para predição, 
o que é esperado quando a evolução futura dos dados segue o mesmo padrão de crescimento que 
foi apresentado durante a fase de treinamento do modelo.

Apesar de excelentes resultados, é importante relembrar que existem mais fatores que afetam 
a evolução dos indicadores avaliados neste estudo do que apenas os seus valores passados 
e uma tendência de evolução. Políticas públicas, outros indicadores sociais e ruídos 
inexplicáveis devem ser levados em conta, e por este motivo o grupo recomenda que para 
trabalhos futuros que foquem no aspecto de uma evolução de longo prazo, como 40 anos ou 
mais, esse problema seja tratado como uma evolução multivariada com séries temporais,
levando em conta outros indicadores e como eles afetam os indicadores analisados aqui.
