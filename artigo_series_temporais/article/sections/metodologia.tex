\section{Metodologia}

A primeira etapa do presente trabalho consistiu na obtenção de um conjunto 
de dados disponibilizados pelo Banco Mundial \cite{Dataset:6}, que é composto por uma 
coletânea de diferentes valores para métricas sociais, socioeconômicas e 
populacionais brasileiras, coletadas entre 1960 e 2020.

Foram escolhidos três indicadores que representam diferentes aspectos do 
desenvolvimento populacional brasileiro e que são adequados para modelagens 
de séries temporais. Isto é, indicadores que apresentam um padrão de crescimento 
característico, consistente e previsível. Os indicadores escolhidos são 
``SP.POP.TOTL'', que representa a população total do Brasil em milhões de 
pessoas, ``SP.RUR.TOTL.ZS'', que representa a porcentagem da população brasileira 
que vive em áreas rurais e ``SP.URB.TOTL.IN.ZS'', que representa a porcentagem 
da população brasileira que vive em centros urbanos. O objetivo do grupo ao 
escolher os indicadores explicitados é demonstrar a aplicação de séries temporais 
em dados populacionais e a modelagem de um preditor capaz de projetar um 
crescimento realista dentro de um período de dez anos.

Para este fim, o conjunto de dados foi separado em conjuntos de treino, 
validação e teste, com o objetivo de predizer os valores para os indicadores 
escolhidos para dez anos no futuro com alto grau de confiabilidade. Para isso,
 foi estabelecida uma taxa de 83\% dos dados sendo usado para treinamento do modelo 
 e 2\% dos dados sendo usados para o conjunto de validação, com 15\% dos dados 
 sobrando para o conjunto de teste do modelo. A escolha por trás dessa alta 
 taxa de treinamento foi motivada principalmente pelos valores observados no 
 crescimento populacional brasileiro serem afetados por incontáveis fatores 
 como políticas públicas, contextos históricos, fenômenos sociais e 
 desenvolvimentos no âmbito socioeconômico brasileiro ao longo de anos, e 
 por isso se faz imperativo treinar o preditor com dados mais recentes possíveis 
 a fim de suprimir o ruído causado por todas estas variáveis. 


Os dados fornecidos para cada indicador são medidos em unidades diferentes, 
relativas ao indicador que está sendo analisado. Na tabela 1 segue um exemplo 
de valores para seis diferentes anos no banco de dados envolvendo os indicadores descritos.


\begin{table}[h]
    \centering
    \begin{tabular}{|c|c|c|c|}
        \hline
        Ano & SP.POP.TOTL & SP.RUR.TOTL.ZS & SP.URB.TOTL.IN.ZS \\
        \hline
        1960 & 72.179235 & 53861 & 46139 \\
        \hline
        1961 & 74.311338 & 52878 & 47122 \\
        \hline
        1962 & 76.514329 & 51901 & 48099 \\
        \hline
        1963 & 78.772647 & 50921 & 49078 \\
        \hline
        1964 & 81.064572 & 49941 & 50059 \\
        \hline
        1965 & 83.373533 & 48963 & 51037 \\
        \hline
    \end{tabular}
    \caption{Amostra de valores dos primeiros 6 anos para os indicadores selecionados}
\end{table}

O presente trabalho analisa os dados obtidos como séries temporais, uma vez que eles 
se encaixam na definição encontrada em “Time Series Analysis: Forecasting and Control” 
\cite{BOX:1}. Neste estudo a ordem de observação dos dados é de extrema relevância, 
pois ela aponta para características e padrões na série e permite com que o preditor consiga 
modelar a evolução dos valores naquela série.

Também conhecida como suavização exponencial, o método de Holt é utilizado para calcular 
previsões em séries temporais que apresentam tendência \cite{Peter:7}. Para se prever valores em Y a 
partir de um conjunto de dados com o método de suavização exponencial de Holt, é necessário 
saber a tendência e o nivelamento da série temporal gerada. Essa expressão é dada por:

\begin{figure}[h]
    \centering
    \begin{equation}
        F(t) = L(t) + T(t) + R
    \end{equation}  
    \caption{Expressão que rege a série temporal}  
\end{figure}

onde $L(t)$ é o nivelamento da série, $T(t)$ representa a inclinação da linha na 
qual os dados estão distribuídos e R é o ruído presente nos dados. O nível da série 
temporal, ou $L(t)$, representa o valor em $Y$ da série temporal em um instante $T$. 
Para estimar esse valor, é necessário utilizar uma Equação de Atualização de Nível, 
que é expressada por: 

\begin{figure}[h]
    \centering
    \begin{equation}
        L(t) = \alpha * (\frac{Y(t)}{S(t)}) + (1 - \alpha) * (L(t-1) + T(t-1))
    \end{equation}
    \caption{Equação de nível da série}
\end{figure}

Na figura 2, é escolhido um valor para alpha $(\alpha)$, o que afeta a representatividade dos 
valores de níveis-base passados. Valores próximos a um reduzem o peso de valores muito antigos, 
e valores próximos a zero dão o mesmo peso para todos os valores presentes na série. 
Dessa forma, o modelo aprende qual será o nível referente a um novo ano que foi inserido 
na série. Também é importante notar que a série apresenta tendência, ou $T(t)$, que é entendida 
como a angulação da linha da série temporal \cite{WIENER:8}. Essa tendência foi identificada 
como sendo de tendência aditiva, visto que os dados se aproximam de uma evolução linear.

A segunda parte do estudo consistiu na obtenção dos dados temporais para os indicadores 
escolhidos e a passagem dessa série para o modelo de Holt utilizando o valor 1.4 para 
alpha a fim de dar um peso maior para valores mais recentes da série, já que é desejável 
que valores observados mais recentementes ditem o sentido de evolução do modelo. 
O valor 1.4 foi descoberto como valor ideal após ajustar o modelo para ele prever 
bem o conjunto de dados de validação.

Os valores previstos para cada ano de teste, que consiste nos anos de 2010 a 2020, 
foram em seguida armazenados e modelados com a biblioteca matplotlib para visualização 
e comparação com os valores reais observados para este período. Para a etapa de avaliação 
do modelo, foram escolhidas as métricas de avaliação \emph{MAE} (\emph{Mean Absolute Error}) e o \emph{R2 Score}. 
Tendo em vista que os indicadores sociais escolhidos são facilmente modeláveis por modelos lineares 
e não apresentam características que dificultam o aprendizado de séries temporais como sazonalidade 
e trends excêntricas, uma variância aceitável é definida pelo grupo na tabela 2.

\begin{table}[h]
    \centering
    \begin{tabular}{|c|c|c|}
        \hline
        Indicador & Valor ideal para \emph{MAE} & Valor ideal para \emph{R2} \\
        \hline
        SP.POP.TOTL & 8 & 0.75 \\ 
        \hline
        SP.RUR.TOTL.ZS & 4 & 0.7 \\
        \hline
        SP.URB.TOTL.IN.ZS & 4 & 0.7 \\
        \hline
    \end{tabular}
    \caption{Resultados considerados ótimos para cada métrica e para cada indicador populacional}
\end{table}